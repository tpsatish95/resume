%% The MIT License (MIT)
%%
%% Copyright (c) 2015 Daniil Belyakov
%%
%% Permission is hereby granted, free of charge, to any person obtaining a copy
%% of this software and associated documentation files (the "Software"), to deal
%% in the Software without restriction, including without limitation the rights
%% to use, copy, modify, merge, publish, distribute, sublicense, and/or sell
%% copies of the Software, and to permit persons to whom the Software is
%% furnished to do so, subject to the following conditions:
%%
%% The above copyright notice and this permission notice shall be included in all
%% copies or substantial portions of the Software.
%%
%% THE SOFTWARE IS PROVIDED "AS IS", WITHOUT WARRANTY OF ANY KIND, EXPRESS OR
%% IMPLIED, INCLUDING BUT NOT LIMITED TO THE WARRANTIES OF MERCHANTABILITY,
%% FITNESS FOR A PARTICULAR PURPOSE AND NONINFRINGEMENT. IN NO EVENT SHALL THE
%% AUTHORS OR COPYRIGHT HOLDERS BE LIABLE FOR ANY CLAIM, DAMAGES OR OTHER
%% LIABILITY, WHETHER IN AN ACTION OF CONTRACT, TORT OR OTHERWISE, ARISING FROM,
%% OUT OF OR IN CONNECTION WITH THE SOFTWARE OR THE USE OR OTHER DEALINGS IN THE
%% SOFTWARE.

% The font could be set to Windows-specific Calibri by using the 'calibri' option
\documentclass[]{mcdowellcv}

% For mathematical symbols
% \usepackage{amsmath}
\usepackage{fontawesome}
\usepackage{geometry}
\usepackage{hyperref}
\usepackage[none]{hyphenat}
\usepackage{microtype}
% \usepackage[T1]{fontenc}
% \usepackage{pslatex}
% \geometry{reset,scale=0.88}

% Set applicant's personal data for header
\name{Satish Palaniappan \linebreak {\small \faPhone \space \href{tel:14124991316}{+1 (412) 499-1316}}}
\address{\href{https://goo.gl/maps/vpb2k4HQ6WLMjiw67}{2720 152nd Ave NE, Unit #561 \linebreak Redmond, WA - 98052.}}
\webpage{}
\contacts{\faEnvelope \space  \href{mailto:tpsatish95@gmail.com}{tpsatish95@gmail.com}  \linebreak \faLinkedin \space \href{https://www.linkedin.com/in/satishpalaniappan}{/in/satishpalaniappan}\linebreak}
\github{\faGithub \space \href{https://github.com/tpsatish95}{/tpsatish95}}


\begin{document}

	% Print the header
	\makeheader
	
	% Print the content

% 	\begin{cvsection}{Employment / Industry Experience}
	\begin{cvsection}{Experience}
	        \begin{cvsubsection}{Software Engineer}{Microsoft}{Jun 2020 – Present}
			\textit{Customer Success Engineering} \hfill \textit{Manager: Curtis Anderson (Principal Software Engineer)}
			\begin{itemize}
            \item Starting work as an applied machine learning engineer under the MARVEL team in the Customer Success Engineering (CSE) org and Experiences + Devices (E+D) engineering group. My role involves building intelligent solutions to improve the Office and Windows product experiences.
			\end{itemize}
		\end{cvsubsection}
	        \begin{cvsubsection}{SDE Intern}{Amazon (AWS)}{Jun 2019 – Aug 2019}
			\textit{HPC Performance Benchmarking Framework} \hfill \textit{Manager: Linda Hedges (Principal SDM)}
			\begin{itemize}
            \item Architected and built an extensible end-to-end automated framework that creates HPC clusters globally across all 50 AWS availability zones; installs and runs various performance benchmarks; and retrieves, parses, stores, searches and visualizes the metrics over time, via an interactive dashboard. This uncovered significant performance degrades in specific AZs. %for certain instance types.
            \item Helped set up the Infosphere team in the High-Performance Computing organization, being the first-ever member.
			\end{itemize}
		\end{cvsubsection}

		\begin{cvsubsection}{Software Engineer}{Qube Cinema Technologies}{Jun 2016 – May 2018}
			\textit{iCount \& Dispatcher} \hfill \textit{Manager: Rajesh Ramachandran (CTO)}
			\begin{itemize}
            \item Designed and developed a 99\% accurate, deep-learned, scalable viewer-demographics mining engine, using convolutional neural networks to extract the count, age, \& gender of the movie watchers from low-light images of a theatre's auditorium. %The resulting deep-learned model had an accuracy of 99\%.
            \item Architected and built a real-time resource allocation and optimization algorithm for making logistical business decisions and maximizing profits intelligently, based on minimum cost flow (transportation) problem and pruned search trees.
            \item Developed a bot for automatically syncing theatre databases across the globe, into one unified format, using Word2Vec.
			\end{itemize}
		\end{cvsubsection}

% 	\end{cvsection}

% \begin{cvsection}{Research Experience}
		\begin{cvsubsection}{Research Assistant}{Institute of Mathematical Sciences}{Dec 2015 - Dec 2017}
		    \textit{\href{https://github.com/tpsatish95/indus-script-ocr}{{\color{blue!70}{Optical Character Recognition on Indus Scripts}}}} \hfill \textit{Advisor: Prof. Ronojoy Adhikari}
			\begin{itemize}
				\item Architected and implemented a deep-learned \href{https://www.youtube.com/watch?v=qPF1oR9yMNY}{\color{blue!70}{\textit{optical character recognition engine}}} that can recognize the 417+ Indus script symbols from images of ancient Harappan civilization artifacts. The symbol classification module has an accuracy of 92\%.%This was built based on GoogLeNet, Transfer Learning, and Selective Search, and it classifies the most frequent symbol - the Jar sign, with an accuracy of 92\%.
				\item Published this work as a research paper titled \href{https://arxiv.org/pdf/1702.00523.pdf}{\color{blue!70}{``\textit{Deep Learning the Indus Script}"}}. News articles covering this work were published in \href{https://www.theverge.com/2017/1/25/14371450/indus-valley-civilization-ancient-seals-symbols-language-algorithms-ai\#EQQA6r}{{\color{blue!70}{\textit{The Verge}}}}, \href{http://www.thehindu.com/sci-tech/science/chennai-team-taps-ai-to-read-indus-script/article17448690.ece}{{\color{blue!70}{\textit{The Hindu}}}}, \href{http://timesofindia.indiatimes.com/city/chennai/app-may-help-decipher-indus-valley-symbols/articleshow/57281369.cms}{{\color{blue!70}{\textit{Times of India}}}}, and \href{http://www.sbs.com.au/yourlanguage/tamil/en/content/app-decipher-ancient-symbols?language=en}{{\color{blue!70}{\textit{SBS Radio - Australia}}}}.
			\end{itemize}
		\end{cvsubsection}
		
		\begin{cvsubsection}{Data Scientist - Intern}{Serendio Inc.}{May 2015 - Jul 2015}
		    \textit{DisKoveror - Text Analytics} \hfill \textit{Manager: Ravi Condamoor (CEO)}
			\begin{itemize}
            	\item Implemented a \href{https://github.com/tpsatish95/Universal-MultiDomain-Sentiment-Classifier}{\color{blue!70}{\textit{universal multi-domain sentiment scorer}}} for text, that supports 36 domains and has an accuracy of 90\%.
                \item Engineered a \href{https://github.com/tpsatish95/Topic-Modeling-Social-Network-Text-Data}{\color{blue!70}{\textit{topic modeling algorithm}}} using hierarchical k-means and semantic word clusters, with an accuracy of 80\%.
                \item Designed an \href{https://github.com/tpsatish95/SocialTextFilter}{\color{blue!70}{\textit{internet-slang text parser}}} that can normalize 6 different artifacts ranging from acronyms to emoticons.
                % \item Serendio's Campus Ambassador at Sri Sivasubramaniya Nadar College of Engg. (SSN CE, affiliated to Anna University).
			\end{itemize}
		\end{cvsubsection}
		
		\begin{cvsubsection}{Research Intern}{Carnegie Mellon University}{Nov 2014 - Dec 2014}
		    \textit{\href{https://github.com/tpsatish95/emotion-detection-from-text}{\color{blue!70}{Text-based Emotion Recognition System}}} \hfill \textit{Advisors: Prof. Bhiksha Raj \& Prof. Rita Singh}
			\begin{itemize}
			 %   \item Built a 90.9\% accurate emotion classifier (7 classes) for text, using histograms built over word2vec word/phrase clusters.
				\item Built a classification model for assigning emotion labels to text data, using histograms built over Word2Vec word/phrase clusters. This model can classify the 7 basic emotions with an accuracy of 90.9\%.
			\end{itemize}
		\end{cvsubsection}
	\end{cvsection}

\begin{cvsection}{Education}
		\begin{cvsubsection}{Masters in Computer Science}{Johns Hopkins University}{May 2020}
			\begin{itemize}
				\item \textit{Teaching Assistant} (Spring '20) for Machine Learning: Deep Learning, under \href{https://samiroid.github.io/}{\color{blue!70}\textit{Prof. Silvio Amir}}.
				\item \textit{Research Assistant} (Spring '19), \href{https://jovo.me/}{\color{blue!70}\textit{Prof. Joshua Vogelstein}}, Research area: \href{https://github.com/tpsatish95/deep-conv-rf}{\color{blue!70}\textit{Stacked Convolutional Random Forests}}.
				\item \textit{Teaching Assistant} (Fall '18) for Object-Oriented Software Engineering, under \href{https://samiroid.github.io/}{\color{blue!70}\textit{Prof. Scott Smith}}.
				\item \textbf{Coursework:} Artificial Intelligence, Machine Learning, Deep Learning, Parallel Programming, Information Retrieval and Web Agents, Neuro Data Design (CGPA: 3.88/4.0).
			\end{itemize}
		\end{cvsubsection}
		\begin{cvsubsection}{Bachelors in Computer Science}{Anna University}{May 2016}
			\begin{itemize}
				\item \textbf{Thesis:} \href{https://github.com/tpsatish95/image-captioning}{\color{blue!70}{\textit{Automated Scenario Description for Images}}} - built an image captioning algorithm using deep learning capable of describing Pokémon battle scenes with natural language descriptions.
				\item \textbf{Scores:} CGPA: 8.56/10, GRE: 324/340 (Quant: 169/170), TOEFL: 116/120.
			\end{itemize}
		\end{cvsubsection}
	\end{cvsection}

    \begin{cvsection}{Skills}
		\begin{cvsubsection}{}{}{}
			\begin{itemize}
				\item \textbf{Languages:} Python, Java, R, C, C++ %, VB.Net
				\item \textbf{Libraries:} PyTorch, Caffe, Keras, Scikit-Learn, Gensim, NLTK, NumPy, SciPy, OpenCV, Matplotlib, Pandas, Flask, Dash
				\item \textbf{Others:} AWS, Docker, DynamoDB, Elasticsearch, Cython, Git, Unix, DevOps
			\end{itemize}
		\end{cvsubsection}
	\end{cvsection}
	
	\begin{cvsection}{Projects}
		\begin{cvsubsection}{}{}{}
			\begin{itemize}
				\item \textbf{\href{https://github.com/neurodata/mgcpy}{\color{blue!70}{\textit{mgcpy \href{https://hyppo.neurodata.io/}{\color{blue!70}{\textit{(hyppo)}}}}}}}: A comprehensive high-dimensional independence and k-sample testing Python package. The code/package has been merged into \href{https://github.com/scipy/scipy/pull/10524}{\color{blue!70}{\textit{SciPy}}} and the \href{https://arxiv.org/abs/1907.02088}{\color{blue!70}{\textit{research paper}}} has been submitted to the Journal of Machine Learning Research (JMLR) and the Journal of Statistical Software.
				\item \textbf{\href{https://www.youtube.com/watch?v=iWw4_Ub2lPw&feature=youtu.be}{\color{blue!70}{\textit{Distributed Panorama Construction from High-Resolution UAV Images Using Public Compute Nodes}}}}, \href{https://www.facebook.com/gtuoffice/videos/1786668221397028/?t=8798}{{\color{blue!70}{\textit{Indian Space Research Organization (ISRO), Smart India Hackathon 2018}}}}. Won the 1\textsuperscript{st} place in this nation-wide hackathon. 
				It was covered by \href{https://www.thehindu.com/todays-paper/tp-national/tp-tamilnadu/5-teams-from-ssn-college-win-prizes/article23461319.ece}{{\color{blue!70}{\textit{The Hindu}}}} \& \href{https://timesofindia.indiatimes.com/city/ahmedabad/six-winners-emerge-from-smart-india-hackathon/articleshow/63561717.cms?utm_source=whatsapp&utm_medium=social&utm_campaign=TOIMobile}{{\color{blue!70}{\textit{Times of India}}}}.
				\item \textbf{\href{https://github.com/tpsatish95/covid19-search-engine}{\color{blue!70}{\textit{COVID-19 Search Engine}}}}: Localized and Personalized Search Engine for keeping track of the dynamic and huge inflow of information during the COVID-19 global pandemic.  Our search engine crawls, aggregates, indexes and searches/retrieves information from local news sources in Baltimore and reports back relevant and personalized results to the user.
				\item \href{https://github.com/tpsatish95/pokemon-vqa}{\color{blue!70}{\textbf{Pokémon VQA}}}: Solves the Visual Question Answering problem in the \textit{Pokémon} domain with an accuracy of 65.9\%.
				\item \textbf{\href{https://www.youtube.com/watch?v=O-L_uSHqQvQ}{\color{blue!70}{\textit{Universally Compatible and Accessible, Software Controlled, Expandable Home Automation Systems}}}}: 
				% Paper: \href{http://research.ijcaonline.org/volume116/number11/pxc3902601.pdf}{\color{blue!70}{``\textit{Home Automation Systems - A Study}"}} in IJCA (\href{https://www.researchgate.net/publication/275338025_Home_Automation_Systems_-_A_Study}{\color{blue!70}{cited 41 times}}). Indian Patent Ref. ID: 5729/CHE/2015.
				This \href{https://github.com/tpsatish95/home-automation-system}{\color{blue!70}{research project}} was funded by the Innovation Center at the \textit{Sri Sivasubramaniya Nadar College of Engineering} (SSN CE, affiliated to Anna University). It was also published as a paper titled: \href{http://research.ijcaonline.org/volume116/number11/pxc3902601.pdf}{\color{blue!70}{``\textit{Home Automation Systems - A Study}"}} in IJCA (\href{https://scholar.google.co.in/citations?user=gNr8v84AAAAJ&hl=en&oi=ao}{\color{blue!70}{cited 45 times}}). Indian Patent Ref. ID: 5729/CHE/2015.
			\end{itemize}
		\end{cvsubsection}
	\end{cvsection}

	\begin{cvsection}{Leadership \& Achievements}
		\begin{cvsubsection}{}{}{}
			\begin{itemize}
				\item \textbf{Merit Scholarship} (Full), for \textit{Excellence in Academics}, SSN College of Engineering, Anna University.
                \item \textbf{Microsoft Research} certified, for proficiency in ``\textit{Design and Analysis of Algorithms}".
                \item\textbf{Government of India}, \textit{Industry Mentor}, for 2 consecutive years at the \textit{Smart India Hackathon} (world's largest).
                \item \textbf{Association for Computing Machinery (ACM)}, \textit{Chairman} (2015-16), \textit{Treasurer \& Tech Lead} (2014-15), SSN CE.
                \item \textbf{Outstanding Student Organizer Award} (2016), \textit{ACM Student Chapter}, SSN CE.
			\end{itemize}
		\end{cvsubsection}
	\end{cvsection}

\end{document}
